%%%%%%%%%%%%%%%%%%%%%%%%%%%%%%%%%%%%%%%%%%%
%%% DOCUMENT PREAMBLE %%%
%This template was adapted from a template by Roza Aceska.
\documentclass[12pt]{report}
\usepackage[english]{babel}
%\usepackage{natbib}
\usepackage{url}
\usepackage[utf8x]{inputenc}
\usepackage{amsmath}
\usepackage{graphicx}
\usepackage{parskip}
\usepackage{fancyhdr}
\usepackage{vmargin}
\usepackage{caption}
\usepackage{subcaption}

\usepackage{tabularx}
\usepackage{xcolor,colortbl}
\newcommand{\notimportant}{\cellcolor{black!10}x} 
\usepackage{hyperref}
\usepackage{cleveref}
\usepackage{float}

\setmarginsrb{3 cm}{2.5 cm}{3 cm}{2.5 cm}{1 cm}{1.5 cm}{1 cm}{1.5 cm}

\title{Report Assignment Part 2}
% Title
\author{}						
% Author
\date{17.07.2020}
% Date

\makeatletter
\let\thetitle\@title
\let\theauthor\@author
\let\thedate\@date
\makeatother

\pagestyle{fancy}
\fancyhf{}
\rhead{\theauthor}
\lhead{\thetitle}
\cfoot{\thepage}
%%%%%%%%%%%%%%%%%%%%%%%%%%%%%%%%%%%%%%%%%%%%
\begin{document}

%%%%%%%%%%%%%%%%%%%%%%%%%%%%%%%%%%%%%%%%%%%%%%%%%%%%%%%%%%%%%%%%%%%%%%%%%%%%%%%%%%%%%%%%%

\begin{titlepage}
	\centering
    \vspace*{0.5 cm}
  \begin{center}    \textsc{\Large   Adavanced Process Mining SS20}\\[2.0 cm]	\end{center}
	\rule{\linewidth}{0.2 mm} \\[0.4 cm]
	{ \huge \bfseries \thetitle}\\
	\rule{\linewidth}{0.2 mm} \\[1.5 cm]
	
  \begin{minipage}{0.48\textwidth}
    \begin{flushleft} \large
      \emph{Submitted To:}\\
      Tobias Brockhoff\\
      Lisa Mannel\\
      Sebastiaan J. van Zelst MSc PhD\\
    \end{flushleft}
  \end{minipage}~
  \begin{minipage}{0.48\textwidth}
    \begin{flushright} \large
			\emph{Submitted By:} \\
      Sezin Maden (354463) \\
      Bruno Machado Pacheco (403532)  \\
      Tom-Hendrik Hülsmann (355773)
		\end{flushright}
	\end{minipage}\\[2 cm]
	
\end{titlepage}

%%%%%%%%%%%%%%%%%%%%%%%%%%%%%%%%%%%%%%%%%%%%%%%%%%%%%%%%%%%%%%%%%%%%%%%%%%%%%%%%%%%%%%%%%

\renewcommand{\thesection}{\arabic{section}}

\section{Q1. Preliminary Analysis}
We begin our analysis by creating two sub-logs of the original event log in a jupyter notebook running pm4py.
By only leaving traces containing an appeal activity in the first sub-log and only leaving traces without an appeal activity for the second sub-log.
This allows us to compare metrics of the original log with the newly created sub-logs:

\begin{table}[H]
\centering
\begin{tabular}{|l|l|l|l|l|}
\hline \textbf{Metric} & \textbf{Original} & \textbf{Appeals} & \textbf{No Appeals} \\
\hline Number of Traces & 150370 & 4513 & 145857\\
\hline Number Trace Variants & 231 & 170 &61\\
\hline Number of Events & 561470 & 29724 & 531746\\
\hline Average Trace Length & 3.73 & 6.59 & 3.65\\
\hline Start of Timespan & 01.01.2000 & 03.01.2000 & 01.01.2000\\
\hline End of Timespan & 18.06.2013 & 14.06.2013 & 18.06.2013\\
\hline Traces with Dublicate Activities (\%)  & 0.0812 & 0.0252 & 0.0843\\
\hline
\end{tabular}
\caption{Basic Metrics of the 3 event logs}
\label{tab:1b}
\end{table}

Since all lifecycle transitions are complete events the number of events and activities are equal. Furthermore, we computed the absolute and relative frequencies of the activities in each log.

\begin{table}[H]
\centering
\begin{tabular}{|l|l|l|l|l|}
\hline \textbf{Activity} & \textbf{Original} & \textbf{Appeals} & \textbf{No Appeals} \\
\hline Add penalty & 79860 & 4357 & 75503\\
\hline Appeal to Judge & 555 & 555 &0\\
\hline Create Fine & 150370 & 4513 & 145857\\
\hline Insert Date Appeal to Prefecture & 4188 & 4146 & 42\\
\hline Insert Fine Notification & 79860 & 4357 & 75503\\
\hline Notify Result Appeal to Offender & 896 & 863 & 33\\
\hline Payment  & 77601 & 903 & 76698\\
\hline Receive Result Appeal from Prefecture & 999 & 964 & 35\\
\hline Send Appeal to Prefecture  & 4141 & 4141 & 0\\
\hline Send Fine  & 103987 & 4513 & 99474\\
\hline Send for Credit Collection & 59013 & 412 & 58601\\
\hline
\end{tabular}
\caption{Absolute Frequencies of Activities in the Event Logs}
\label{tab:1c_absolut}
\end{table}

\begin{table}[H]
\centering
\begin{tabular}{|l|l|l|l|l|}
\hline \textbf{Activity} & \textbf{Original} & \textbf{Appeals} & \textbf{No Appeals} \\
\hline Add penalty & 0.142234 & 0.146582 & 0.141991\\
\hline Appeal to Judge & 0.000988 & 0.018672 &0\\
\hline Create Fine & 0.267815 & 0.151830 & 0.274298\\
\hline Insert Date Appeal to Prefecture & 0.007459 & 0.139483 & 0.000079\\
\hline Insert Fine Notification & 0.142234 & 0.146582 & 0.141991\\
\hline Notify Result Appeal to Offender & 0.001596 & 0.029034 & 0.000062\\
\hline Payment  & 0.138210 & 0.030379 & 0.144238\\
\hline Receive Result Appeal from Prefecture & 0.001779 & 0.032432 & 0.000066\\
\hline Send Appeal to Prefecture  & 0.007375 & 0.139315 & 0\\
\hline Send Fine  & 0.185205 & 0.151830 & 0.187071\\
\hline Send for Credit Collection & 0.105104 & 0.013861 & 0.110205\\
\hline
\end{tabular}
\caption{Relative Frequencies of Activities in the Event Logs}
\label{tab:1c_relative}
\end{table}

All traces of the original log begin with a 'Create Fine' activity. Therefore, the fraction of traces starting with 'Create Fine' is $1$ and $0$ for all other activities. Of course, this is also the case for both sub-logs. \\
The following activities are the only activities that occur as end activities of traces:

\begin{table}[H]
\centering
\begin{tabular}{|l|l|l|l|l|}
\hline \textbf{Activity} & \textbf{Original} & \textbf{Appeals} & \textbf{No Appeals} \\
\hline Payment  & 0.446904 & 0.154443 & 0.455953\\
\hline Send for Credit Collection & 0.392346 & 0.087747 & 0.401770\\
\hline Send Fine  & 0.138026 & 0.001551 & 0.142249\\
\hline Send Appeal to Prefecture  & 0.020908 & 0.696654 & 0\\
\hline Appeal to Judge & 0.000891 & 0.029692 &0\\
\hline Notify Result Appeal to Offender & 0.000572 & 0.018170 & 0.000027\\
\hline Receive Result Appeal from Prefecture & 0.000352 & 0.011744 & 0\\
\hline
\end{tabular}
\caption{Fraction of Traces that end with the Corresponding Activity}
\label{tab:1c_end}
\end{table}

Using Tabel \ref{tab:1b} we can compare the tree logs on a basic level. As expected, the majority of
cases do not contain a appeal event. Only around $3\%$ of traces have an appeal. But since the average trace length is
significantly longer in the cases with an appeal, the events in the log with appeals are around $5.3\%$ of the original log.
With 60 unique traces the log with traces is responsible for a disproportionately high trace variety. The original and the log
without appeals have data starting on the 01.01.2000 until the 18.06.2013. The timespan of the log with appeals is very similar
only starting a few days later and ending a few days earlier. This is to be expected due to the relatively low amount of traces
with an appeal. Interestingly, the log with appeals contains far less traces with duplicate activities than the other two logs. \\

Some interesting insights can be gained by looking at the absolute and relative frequencies of specific activities in Table \ref{tab:1c_absolut} and Table \ref{tab:1c_relative}.
First of all, we can note that there are no appearances of the activities 'Appeal to Judge' or 'Send Appeal to Prefecture' in the log without appeals.
There are still some activities that should only be in the log with appeals such as 'Notify Result Appeal to Offender' but these occurrences can
be considered noise and ignored. An open question that could point to a possible problem is the high amount of traces containing 'Send for Credit Collection' in the log without appeals. Especially, when keeping in mind that this activity is batched together. It is very likely that the average
case duration could be reduced if there were less cases where the fine has to be collected externally. The same thing can also be observed in Table \ref{tab:1c_end}. Furthermore, in the case without appeals, there are many traces ending with a 'Send Fine' event which points to a high number of unpaid fines. Further analysis should investigate if this suspicion is true and what can be done against it.

\section{Q2. Discovery and Conformance}

\section{Q3. Attribute Analysis and Compliance}

\section{Q4. Performance}

\section{Q5. Summary and Conclusion}

\section*{Appendix}

\end{document}

